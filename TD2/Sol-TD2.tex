\documentclass[12pt]{article}

\usepackage[top=2cm,bottom=2cm,left=2cm,%
right=2cm]{geometry}

\usepackage[french]{babel}
\usepackage[T1]{fontenc}
\usepackage{amsmath}
\usepackage{amssymb}
\usepackage[dvipsnames]{xcolor}
\usepackage{mathpartir}



\begin{document}
\def\tobar{\mathrel{\mkern3mu \vcenter{\hbox{$\scriptscriptstyle+$}}\mkern-12mu{\to}}}




\title{\textbf{TD2: Correction}}
\author{Félix Martins-Ducasse}
\date{}
\maketitle

\section{Relations définies (co)inductivement et opérateurs de clôture}\textbf{Rappels et notation.} Dans ce qui suit, $R$, $S$ et $T$ désignent des endorelations d'un ensemble $A$ quelconque, et $F$ et $G$ désignent des endofonctions croissantes de $\textbf{Rel}(A,A)$. On rappelle que $F^\mu$  est la fonction qui envoie $R$ dans $\mu X. R \cup F(X)$, tandis que $\mu G$ est la relation $\mu X.G(X)$.

La notation $R^n$ désigne l'endorelation de $A$ définie par récurrence sur n par $R^0 = Id_A$ et $R^{n+1} = R^n;R$. La notation $F^n$ désigne l'endofonction croissante définie par récurrence sur $n$ par $F^0(X) = X$ et $F^{n+1}(X) = F(F^n(X))$

On définit une relation d'ordre (réfléxive, transitive et antisymétrique) entre endofonctions croissantes de $\textbf{Rel}(A,A)$ en posant $F\leq G$ lorsque pour toute relation $R$ dans $\textbf{Rel}(A,A)$, on a $F(R) \subseteq  G(R)$

\subsection*{1.1 Relations closes}
\subsection*{Exercice 1.1 (*)}
\subsubsection*{Énoncé}
Montrer que si $F(R_1;R_2) \subseteq F(R_1);F(R_2)$ pour toutes endorelations $R_1$ et $R_2$ de $A$, alors la composition de deux relation $F$-close est $F$-close.

\subsubsection*{Correction}
But : $\forall R_1, R_2$ Si $F(R_1;R_2) \subseteq F(R_1);F(R_2)$ alors $\forall A,B \ F$-closes\   $(A;B) \  F$-close

Soient $S_1,S_2 : A\tobar A$ telles que 

$F(S_i) \subseteq S_i \quad (i \in {1,2})$

On a 

\begin{align*}
    F(S_1;S_2) &\subseteq F(S_1);F(S_2)&&\text{Par hyp sur F}\\
               &\subseteq S_1 ; F(S_2)&&\text{Car $S_1$ est $F$-close et ; croissante}\\ 
               &\subseteq S_1 ; S_2 &&\text{idem}
\end{align*}

\vspace{1cm}

\subsection*{Exercice 1.2 (*)}
\subsubsection*{Énoncé}
Comme nous l'avons vu en cours. l'utilisation des plus petis points préfixes permet de définir de façon uniforme la clôture d'une relation sous une certaine propriété (transitivité, reflexivité, symétrie, etc.). Cependant, il est possible dans certains cas de définir ces clôtures d'une façon plus maniable. Chacune des équations ci-dessous propose une telle définition; démontrer-les.

\begin{align}
    R^? &:= \mu X.R \cup Id_A = R \cup Id_A\\ 
    R^\leftrightarrow &:= \mu X.R \cup X^\circ = R \cup R^\circ\\
    R^+ &:= \mu X.X;X = \bigcup_{n \geq 1}{R^n}
\end{align}

\subsubsection*{Corrigé}
(1)
$R \cup Id_A$ ne dépend pas de $X$

\textbf{Lemme}: si $F$ est constante, c'est-à-dire $\forall R_1,R_2, F(R_1) = F(R_2)$ alors $\mu F = F(\emptyset)$


On a 

\begin{align*}
    \mu F &= F(\emptyset)\\ 
    \iff F(\mu F) &= F(\emptyset) && \text{vrai car $F$ est constante}
\end{align*}


(2)

Il suffit de montrer que $R\cup R^\circ$ est le plus petit point préfixe de $F(X) = R\cup X^\circ$


Montrons que $R\cup R^\circ$ est un point préfixe 
\begin{align*}
 F(R\cup R^\circ) &= R \cup (R\cup R^\circ)^\circ \\ 
                  &= R \cup R^\circ \cup R^{\circ\circ}\\ 
                  &= R \cup R^\circ
\end{align*}

Soit $S$ un point préfixe 

\begin{align*}
    F(S) &\subseteq S\\ 
    R \cup S^\circ &\subseteq S\\ 
    R \subseteq S \quad &\text{et} \quad S^\circ \subseteq S \\ 
    R^\circ {\subseteq} S^\circ \quad &\text{et} \quad R^\circ \subseteq S\\ 
    R \cup R^\circ &\subseteq S &&\square
\end{align*}

$R\cup R^\circ$ est donc le plus petit point préfixe.

\vspace{5cm}


(3)

$R' = \bigcup_{n \geq 1}{ R^n}$

Montrons que c'est un plus petit point préfixe de $G(X) = R \cup X^2$

Montrons que c'est un point préfixe, il suffit de montrer $R \subseteq R'$ et $R'^2 \subseteq R'$

\[R = R^1 \subseteq \bigcup_{n \geq 1}{R^n} = R'\]

Donc $ R \subseteq R'$

\begin{align*}
    R';R'&= (\bigcup_{n \geq 1}{R^n});(\bigcup_{m \geq 1}{R^m})\\
         &= \bigcup_{n \geq 1}{}\bigcup_{m \geq 1}{R^n;R^m}\\
        &= \bigcup_{n \geq 1}{}\bigcup_{m \geq 1}{R^n+m}\\ 
        &= \bigcup_{p \geq 2}{R^p} \\
        &\subseteq \bigcup_{p \geq 1}{R^p}\\
        &= R' && \square
\end{align*}

On a donc $R'$ qui est un point préfixe.
\\

Soit $S$ tel que $R\cup S^2 \subseteq S$, Montrons que $R' \subseteq S$, c'est-à-dire :

\begin{align*}
    \bigcup_{n \geq 1}{R^n} \subseteq S
\end{align*}

On va procéder par récurrence et montrer que $\forall n \geq 1, R^n \subseteq S$
\begin{description}
\item[Cas de base $n=1$:]
    \[ R^1 = R \subseteq R \cup S^2 \subseteq S \]

\item[Cas $n = m + 1$]
    \begin{align*}
        R^{m+1} &= R^m ; R\\ 
                &\subseteq S ;R && \text{par Hyp Ind}\\
                &\subseteq S;S && \text{par Hyp Ind}\\ 
                &\subseteq S && \text{par Hyp Ind } \quad\square
    \end{align*}
\end{description}

\vspace{4cm}
\subsection*{Exercice 1.3}
\subsubsection*{Énoncé}

On dit que $F\ preserve\ les\ unions\ croissantes$ si, pour toute famille d'ensembles $(A_n)_{n \geq 0}$ telle que $A_n \subseteq A_{n+1}$ pour tout $ n \geq 0$, on a $F(\bigcup_{n \geq 0}{A_n}) = \bigcup_{n \geq 0}{F(A_n)}$.

Montrer que si $F$ préserve les unions croissantes alors : 

\[\mu X.F(X) = \bigcup_{n \geq 0}{F^n(\emptyset)} \]

\subsubsection*{Corrigé} 

Soit $F : \textbf{Rel}(A,A) \rightarrow \textbf{Rel}(A,A) $ croissante

Montrer que 

\[ \mu F = \bigcup_{n \geq 0}{F^n (\emptyset)}\]

dès lors que $F$ préserve les unions croissantes. 

\[(R_n : A \rightarrow A)_{n \geq 0} \ \text{tel que} \ \forall n \geq 0, R_n \subseteq R_{n+1}\]

$F$ préserve les union croissantes $\overset{\text{def}}{\iff} \forall (R_n)_{n \geq 0}$  croissante on a : 

\[ \bigcup_{n \geq 0 }{F(R_n)} = F(\bigcup_{n \geq 0}{R_n})\]


On va montrer que $\bigcup_{n \geq 0 }{F^n(\emptyset)}$ est le plus petit point préfixe de $F$\\

Montrer que c'est un point préfixe : 

On remarque que $(F^n(\emptyset))_{n \geq 0}$ est croissante ce qu'on démontre par induction sur $n$ : 

\[\forall n \quad F^n(\emptyset) \subseteq F^{n+1}(\emptyset)\]


\begin{description}
\item[Cas n = 0 :]
    \[F^\circ (\emptyset) = \emptyset \subseteq F^1(\emptyset)\]
\item[Cas n = m + 1]
    \begin{align*}
        F^{m+1}{\emptyset} &= F(F^m(\emptyset))\\
                           &\subseteq F(F^{m+1}(\emptyset)) && \text{par H.I. et $F$ croissante}\\
                           &= F^{m+2}(\emptyset)\\ 
                           &= F^{n+1}(\emptyset) && \square
    \end{align*}
\end{description}

Soit $S$ un point préfixe\\

Montrons que $\bigcup_{n \geq 0}{F^n(\emptyset)} \subseteq S$\\ 

Il suffit de montrer que $\forall n \geq 0, F^n(\emptyset) \subseteq S$\\ 

On va procéder par induction sur $n$ :\\ 


\begin{description}
    \item[Cas n=0]
    \[F^\circ(\emptyset) = \emptyset \subseteq S\] 
\item[Cas n = m + 1]
    \begin{align*}
        F^{m+1}(\emptyset) &= F(F^m(\emptyset))\\
                           &\subseteq F(S) && \text{par H.I. et $F$ croissante}\\ 
                           &\subseteq S && \square
    \end{align*}
\end{description}

\subsection*{Exercice 1.4}
\subsubsection*{Énoncé}

L\'opération $\mu$ associe à une endofonction croissante des endorelations
de $A$ son plus petit point préfixe, qui est une endorelation de $A$. Puisque ce plus petit
point préfixe existe toujours et est unique, on peut voir $\mu$ comme une fonction : 

$$\mu : (\textbf{Rel}(A,A) \rightarrow_{cr} \textbf{Rel}(A,A)) \rightarrow \textbf{Rel}(A,A)$$\\

où la notation $X \rightarrow_{cr} Y$ désigne les fonctions  croissantes de $X$ dans $Y$. Montre que $\mu$  est elle même croissante.

\subsubsection*{Corrigé}

But : $\mu$ croissante càd : \\

Soit $S_1,S_2 \subseteq \textbf{Rel}(A,A) \rightarrow_{cr} \textbf{Rel}(A,A)$\\

On veut : $S_1 \subseteq S_2 \Rightarrow \mu S_1 \subseteq \mu S_2 $ \\

Par définition on sait que :

$$ S_i(\mu S_i) \subseteq \mu S_i \quad i \in \{1,2\}$$\\ 

Et que 

$$\forall X \in \textbf{Rel}(A,A), \quad S_1(X) \subseteq S_2(X)$$

\begin{align*}
    S_1(\mu S_2) &\subseteq S_2(\mu S_2)\\
    S_1(\mu S_2) &\subseteq \mu S_2 && \square
\end{align*}

$\mu S_2$ est un point préfixe de $S_1 \Rightarrow \mu S_1 \subseteq \mu S_2$

\subsection*{Exercice 1.5}
\subsubsection*{Énoncé}
Dans cet exercice, on s’intéresse à certaines conditions suffisantes pour
que deux fonctions aient le plus petit point fixe. Ces conditions peuvent être utiles pour
donner des caractérisations alternatives d’un plus petit point fixe.
\begin{enumerate}
    \item Montrer que $\mu GF = G\mu FG$.
    \item Montrer que $\mu F^2 = \mu F$.
\end{enumerate}

\subsubsection*{Corrigé}
1. $\mu GF = G\mu FG$.\\

On va démonter l'inclusion dans les deux sens.\\

Montrer que $\mu GF\subseteq G\mu FG$

\begin{align*}
    &\mu GF \subseteq G\mu FG\\
    &\Longleftarrow GFG\mu FG \subseteq G\mu FG\\
    &\iff G\mu FG \subseteq && \square
\end{align*}

Nous appelons le résultat ci-dessus \textbf{Lemme A}\\

Montrer que $G\mu FG \subseteq \mu GF$

\begin{align*}
    &G\mu FG \subseteq \mu GF \\ 
    &\iff G\mu FG \subseteq GF\mu GF\\ 
    &\Longleftarrow \mu FG \subseteq F\mu GF\\
    &\Longleftarrow \text{Lemme A} && \square
\end{align*}


\subsection*{Exercice 1.6}
\subsubsection*{Énoncé}
Notons $\omega + 1$ l'ensemble $\{0,1,2,...,\omega\}$ muni de la relation d'ordre canonique, c'est-à-dire héritée des entiers et telle que $n  < \omega$ pour tout $n$ entier. On considère l'endofonction $F$ des parties de $\omega + 1$ donnée par :

\[
    F(X)
    \begin{cases}
        \{n \mid n \leq \max(X) + 1 \} &\text{Si $X$ est fini et $\omega \notin X$}\\
        \omega + 1 &\text{sinon.}
    \end{cases}
\]

\begin{enumerate}
    \item Montrer que $F$ est croissante.
    \item Donner le plus petit point préfixe de $F$.
    \item Donner les trois premiers termes de la suite d'ensembles $(F^n(\emptyset))_{n\geq 0}, \text{puis} \bigcup_{n \geq 0}$
    \item Que conclure des questions précédentes au sujet de ls préservation des unions croissantes par $F$ ?
\end{enumerate}

\subsubsection*{Corrigé}

1. $F$ est croissante $\iff \forall X,Y \subseteq \omega + 1 \quad X\subseteq Y \Rightarrow F(X) \subseteq F(Y) $\\

Soit $X\subseteq Y$ des parties de $\omega + 1$\\ 

On raisonne par cas sur $X$ :

\begin{enumerate}
    \item $X$ est fini et $\omega \notin X$\\
        On raisonne par cas sur $Y$ :
        \begin{enumerate}
            \item $Y$ fini et $\omega \notin Y$:
                \begin{align*}
                    X \subseteq Y & \Rightarrow \max(X)\leq \max(Y)\\
                                  & \Rightarrow \{n \mid n \leq \max(X) + 1 \} \subseteq \{m \mid m \leq \max(Y) + 1 \}\\ 
                                  & \iff F(X) \subseteq F(Y) && \square
                \end{align*}
            \item Cas $Y$ infini ou $\omega \in  Y$
                \[F(Y) = \omega + 1 \supseteq F(X) \]
        \end{enumerate}
    \item Cas $X$ infini ou $\omega \in X$ :

        alors $Y$ infini ou $\omega \in Y$
        et donc $F(Y) = \omega + 1 = F(X)$
\end{enumerate}

On a donc montrer que $F$ est croissante.\\

2. On sait que le plus petit point préfixe est un point fixe,  c'est-à-dire :
\[F(\mu F) = \mu F\]

Si $X$ est fini et tq $\omega \notin X$, on a $X \subsetneq F(X)$, donc le seul point fixe est : $\mu F = \omega + 1$

Seul possibilité car $\max(\emptyset) = 0$\\ 

3. Premiers termes de la suite d'ensembles $(F^n(\emptyset))_{n\geq 0}$ :

\begin{align*}
    &F^0(\emptyset) = \emptyset\\
    &F^1(\emptyset) = F(\emptyset) = \{0\}\\
    &F^2(\emptyset) = F(F(\emptyset)) = F(\{0\}) = \{0,1\}\\ 
    &\dots
\end{align*}

On a donc :

\[\bigcup_{n \geq 0}{F^n(\emptyset)} = \{n \in \mathbb{N}\}\]

4.Pour déterminer si $F$ préserve les unions de chaînes croissantes, c'est-à-dire si pour toute chaîne $(X_n)_{n \geq 0}$, on a :
\[ F\left( \bigcup_{n \geq 0} X_n \right) = \bigcup_{n \geq 0} F(X_n) \]

Considérons la chaîne croissante définie à la question précédente : $X_n = F^n(\emptyset)$.\\ 

On a donc pour le membre gauche :
\[ F(\mathbb{N}) = \omega + 1\]

Et pour le membre droit :

    \[ \bigcup_{n \geq 0} F(X_n) = \bigcup_{n \geq 0} X_{n+1} = \mathbb{N} \]

    \textbf{Conclusion :}
Comme $\omega + 1 \neq \mathbb{N}$, nous avons :
\[ F\left( \bigcup_{n \geq 0} X_n \right) \neq \bigcup_{n \geq 0} F(X_n) \]

La fonction $F$ ne préserve pas les unions croissantes. 

\subsection*{Exercice 1.7 (*)}

\subsubsection*{Énoncé}
Montrer que $F^{\mu}$ est croissante, inflationnaire et idempotente.

\subsubsection*{Corrigé}

\begin{align*}
    &F^\mu (R) := \mu X. R \cup F(X)\\ 
    1.\quad &F^\mu \text{ croissante} \iff R\subseteq S \Rightarrow F^\mu (R) \subseteq F^\mu (S)\\ 
    2. \quad &F^\mu \text{ inflationnaire} \iff R \subseteq F^\mu(R)\\ 
    3. \quad &F^\mu \text{ idempotente} \iff F^\mu(F^\mu(R)) = F^\mu(R)
\end{align*}

1. Soient $R\subseteq S$ \\

On a: 
\begin{align*}
    &F^\mu(R) \subseteq F^\mu(S)\\ 
    \iff &\mu X.R\cup F(X) \subseteq F^\mu(S)\\ 
    \Leftarrow &R\cup F(F^\mu(S) \subseteq F^\mu(S)\\ 
    \Leftarrow &R \underset{\text{A}}{\subseteq} F^\mu(S) \quad \text{et} \quad F(F^\mu(S)) \underset{\text{B}}{\subseteq} F^\mu (S)
\end{align*}

A :

\begin{align*}
    &R \subseteq F^\mu(S) = S \cup F(F^\mu(S))\\ 
    & \Leftarrow R \subseteq S
\end{align*}

\vspace{1cm}
B :

\begin{align*}
    &F(F^\mu(S)) \subseteq F^\mu(S)\\ 
    &\iff F(F^\mu(S)) \subseteq S \cup F(F^\mu(S)) && \square
\end{align*}


2. On a : 
$$F^\mu(R) = R \cup F(F^\mu(R)) \supseteq R \quad \square$$

3. Comme la fonction $F^\mu$ est inflationnaire il suffit de :
\begin{align*}
    \text{Mq  } &F^\mu(\textcolor{blue}{F^\mu(R)}) \subseteq \textcolor{red}{F^\mu(R)} \qquad\overset{def}{\iff} \qquad \mu X.\textcolor{blue}{F^\mu(R)} \cup F(\textcolor{red}{X}) \subseteq \textcolor{red}{F^\mu(R)}\\ 
    \Leftarrow &\textcolor{blue}{F^\mu(R)} \cup \textcolor{red}{F(F^\mu(R))} \subseteq F^\mu (R) && (1)\\ 
    \Leftarrow &F(F^\mu(R)) \subseteq F^\mu(R)\\ 
    \iff &F(F^\mu(R)) \subseteq R \cup F(F^\mu (R)) && \square
\end{align*}

(1) On démontre ça car : 
\begin{mathpar}
\inferrule[]
    { F(R) \subseteq R }
    { \mu F \subseteq R} 
\end{mathpar}

\end{document}
